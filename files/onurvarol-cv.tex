%%%%%%%%%%%%%%%%%%%%%%%%%%%%%%%%%%%%%%%%%%%%%%%%%%%%%%%%%%%%%%%%%%%%%%%%
%%%%%%%%%%%%%%%%%%%%%% Simple LaTeX CV Template %%%%%%%%%%%%%%%%%%%%%%%%
%%%%%%%%%%%%%%%%%%%%%%%%%%%%%%%%%%%%%%%%%%%%%%%%%%%%%%%%%%%%%%%%%%%%%%%%

%%%%%%%%%%%%%%%%%%%%%%%%%%%%%%%%%%%%%%%%%%%%%%%%%%%%%%%%%%%%%%%%%%%%%%%%
%% NOTE: If you find that it says                                     %%
%%                                                                    %%
%%                           1 of ??                                  %%
%%                                                                    %%
%% at the bottom of your first page, this means that the AUX file     %%
%% was not available when you ran LaTeX on this source. Simply RERUN  %%
%% LaTeX to get the ``??'' replaced with the number of the last page  %%
%% of the document. The AUX file will be generated on the first run   %%
%% of LaTeX and used on the second run to fill in all of the          %%
%% references.                                                        %%
%%%%%%%%%%%%%%%%%%%%%%%%%%%%%%%%%%%%%%%%%%%%%%%%%%%%%%%%%%%%%%%%%%%%%%%%

%%%%%%%%%%%%%%%%%%%%%%%%%%%% Document Setup %%%%%%%%%%%%%%%%%%%%%%%%%%%%

% Don't like 10pt? Try 11pt or 12pt
\documentclass[10pt]{article}


% This is a helpful package that puts math inside length specifications
\usepackage{calc}
% Layout: Puts the section titles on left side of page
\reversemarginpar

%
%         PAPER SIZE, PAGE NUMBER, AND DOCUMENT LAYOUT NOTES:
%
% The next \usepackage line changes the layout for CV style section
% headings as marginal notes. It also sets up the paper size as either
% letter or A4. By default, letter was used. If A4 paper is desired,
% comment out the letterpaper lines and uncomment the a4paper lines.
%
% As you can see, the margin widths and section title widths can be
% easily adjusted.
%
% ALSO: Notice that the includefoot option can be commented OUT in order
% to put the PAGE NUMBER *IN* the bottom margin. This will make the
% effective text area larger.
%
% IF YOU WISH TO REMOVE THE ``of LASTPAGE'' next to each page number,
% see the note about the +LP and -LP lines below. Comment out the +LP
% and uncomment the -LP.
%
% IF YOU WISH TO REMOVE PAGE NUMBERS, be sure that the includefoot line
% is uncommented and ALSO uncomment the \pagestyle{empty} a few lines
% below.
%

%% Use these lines for letter-sized paper
\usepackage[paper=letterpaper,
            includefoot, % Uncomment to put page number above margin
            marginparwidth=1.2in,     % Length of section titles
            marginparsep=.05in,       % Space between titles and text
            margin=0.6in,               % 1 inch margins
            includemp]{geometry}

%% Use these lines for A4-sized paper
%\usepackage[paper=a4paper,
%            %includefoot, % Uncomment to put page number above margin
%            marginparwidth=30.5mm,    % Length of section titles
%            marginparsep=1.5mm,       % Space between titles and text
%            margin=25mm,              % 25mm margins
%            includemp]{geometry}

%% More layout: Get rid of indenting throughout entire document
\setlength{\parindent}{0 in}

%% This gives us fun enumeration environments. compactitem will be nice.
\usepackage{paralist}
%\usepackage{enumitem}

%% Reference the last page in the page number
%
% NOTE: comment the +LP line and uncomment the -LP line to have page
%       numbers without the ``of ##'' last page reference)
%
% NOTE: uncomment the \pagestyle{empty} line to get rid of all page
%       numbers (make sure includefoot is commented out above)
%
\usepackage{fancyhdr,lastpage}
\pagestyle{fancy}
\pagestyle{empty}      % Uncomment this to get rid of page numbers
\fancyhf{}\renewcommand{\headrulewidth}{0pt}
\fancyfootoffset{\marginparsep+\marginparwidth}
\newlength{\footpageshift}
\setlength{\footpageshift}
          {0.5\textwidth+0.5\marginparsep+0.5\marginparwidth-2in}
\lfoot{\hspace{\footpageshift}%
       \parbox{2in}{\, \hfill %
                    \arabic{page} of \protect\pageref*{LastPage} % +LP
%                    \arabic{page}                               % -LP
                    \hfill \,}}

% Finally, give us PDF bookmarks
\usepackage{color,hyperref}
\definecolor{darkblue}{rgb}{0.0,0.0,0.3}
\hypersetup{colorlinks,breaklinks,
            linkcolor=darkblue,urlcolor=darkblue,
            anchorcolor=darkblue,citecolor=darkblue}

%%%%%%%%%%%%%%%%%%%%%%%% End Document Setup %%%%%%%%%%%%%%%%%%%%%%%%%%%%


%%%%%%%%%%%%%%%%%%%%%%%%%%% Helper Commands %%%%%%%%%%%%%%%%%%%%%%%%%%%%

% The title (name) with a horizontal rule under it
%
% Usage: \makeheading{name}
%
% Place at top of document. It should be the first thing.
\newcommand{\makeheading}[1]%
        {\hspace*{-\marginparsep minus \marginparwidth}%
         \begin{minipage}[t]{\textwidth+\marginparwidth+\marginparsep}%
                {\large \bfseries #1}\\[-0.15\baselineskip]%
                 \rule{\columnwidth}{1pt}%
         \end{minipage}}

% The section headings
%
% Usage: \section{section name}
%
% Follow this section IMMEDIATELY with the first line of the section
% text. Do not put whitespace in between. That is, do this:
%
%       \section{My Information}
%       Here is my information.
%
% and NOT this:
%
%       \section{My Information}
%
%       Here is my information.
%
% Otherwise the top of the section header will not line up with the top
% of the section. Of course, using a single comment character (%) on
% empty lines allows for the function of the first example with the
% readability of the second example.
\renewcommand{\section}[2]%
        {\pagebreak[2]\vspace{1.3\baselineskip}%
         \phantomsection\addcontentsline{toc}{section}{#1}%
         \hspace{0in}%
         \marginpar{
         \raggedright \scshape #1}#2}

% An itemize-style list with lots of space between items
\newenvironment{outerlist}[1][\enskip\textbullet]%
        {\begin{itemize}[#1]}{\end{itemize}%
         \vspace{-.6\baselineskip}}

% An environment IDENTICAL to outerlist that has better pre-list spacing
% when used as the first thing in a \section
\newenvironment{lonelist}[1][\enskip\textbullet]%
        {\vspace{-\baselineskip}\begin{list}{#1}{%
        \setlength{\partopsep}{0pt}%
        \setlength{\topsep}{0pt}}}
        {\end{list}\vspace{-.6\baselineskip}}

% An itemize-style list with little space between items
\newenvironment{innerlist}[1][\enskip\textbullet]%
        {\begin{compactitem}[#1]}{\end{compactitem}}

\catcode240=13 \def �{\u g}
\catcode231=13 \def �{\c c}
\catcode246=13 \def �{\"o}
\catcode254=13 \def �{\c s}
\catcode252=13 \def �{\"u}
\catcode253=13 \def �{{\i}}
\catcode221=13 \def �{\.I}
\catcode199=13 \def �{\c C}
\catcode208=13 \def �{\u G}
\catcode214=13 \def �{\"O}
\catcode222=13 \def �{\c S}
\catcode220=13 \def �{\"U}


% To add some paragraph space between lines.
% This also tells LaTeX to preferably break a page on one of these gaps
% if there is a needed pagebreak nearby.
\newcommand{\blankline}{\quad\pagebreak[2]}
\newlength{\rcollength}\setlength{\rcollength}{1.5in}%

\usepackage{graphicx}
\usepackage{scalerel}
\def\emoji#1{\scalerel*{\includegraphics{#1}}{B}}

%%%%%%%%%%%%%%%%%%%%%%%% End Helper Commands %%%%%%%%%%%%%%%%%%%%%%%%%%%

%%%%%%%%%%%%%%%%%%%%%%%%% Begin CV Document %%%%%%%%%%%%%%%%%%%%%%%%%%%%

\begin{document}

\begin{center}
\begin{table}
	\begin{tabular*}{\linewidth}{@{\extracolsep{\fill}}lr}
		\Large{Onur VAROL} & \\
		
		\normalsize{\href{http://www.onurvarol.com}{www.onurvarol.com}} & 
		\normalsize{ \href{http://www.barabasilab.com/}{Center for Complex Network Research} at \href{https://www.networkscienceinstitute.org/}{Northeastern University}}\\
		
		\normalsize{\href{mailto:ovarol@northeastern.edu}{ovarol@northeastern.edu}} & \normalsize{177 Huntington Ave. 11th floor Boston, MA 02115}\\
	\end{tabular*}
\end{table}
\end{center}
\vspace{-10mm}
\hrulefill

% NOTE: Mind where the & separators and \\ breaks are in the following
%\begin{center}{\textbf{Onur VAROL}}\\
%\vspace{5mm}
%\normalsize{e-mail: \href{mailto:ovarol@indiana.edu}{ovarol@indiana.edu}} \\
%\normalsize{website: \href{http://www.onurvarol.com}{www.onurvarol.com}} \\
%\normalsize{GSM: +1 812 369 0513} \\
%\normalsize{ \href{http://www.soic.indiana.edu/}%
%     {School of Informatics and Computing} 
%\href{http://www.iu.edu/}{Indiana University}}\\
%\normalsize{919 E. 10th St, Bloomington IN, 47408, USA}
%\end{center}

%       table.
%
% ALSO: \rcollength is the width of the right column of the table
%       (adjust it to your liking; default is 1.85in).
%

%
%\begin{tabular}[t] {@{}p{\textwidth-\rcollength}p{\rcollength}}

%\section{Objective} 
%To pursue an academic career in the fields of Complex Systems and Network Theory. Especially using statistical mechanics and machine learning approaches.  

\section{Research Interests}
Data Science, Computational Social Science, Machine Learning, Network Science, Complex Systems

\section{Appointments}
\href{https://www.barabasilab.com/}{\textbf{Center for Complex Network Research}}, Northeastern University, Boston, Massachusetts, USA
\begin{innerlist}
	\item[] Postdoctoral Research Associate (July 2017, ---)
\end{innerlist}

\textbf{Indiana University}, Bloomington, Indiana, USA
\begin{innerlist}
	\item[] Teaching Assistant (December 2017, May 2017)
	\item[] Research Assistant (August 2012, December 2017)
\end{innerlist}

%%%%%% EDUCATION %%%%%%
\section{Education}
\href{http://www.iu.edu/}{\textbf{Indiana University}},
Bloomington, Indiana, USA
\begin{innerlist}
\item[] Ph.D.,
        \href{http://www.soic.indiana.edu/}
             {Informatics, Complex Systems Track} and
             minor in Statistics, (June 2017) \\
        Dissertation: Analyzing Social Big Data to Study Online Discourse and its Manipulation\\
        Awarded for University Distinguished Ph.D. Dissertation Award for 2018 in the Social Sciences\\
        Committee: 
        		\href{http://cnets.indiana.edu/fil/}{Filippo Menczer}, 
        		\href{http://cnets.indiana.edu/aflammin/}{Alessandro Flammini}, 
		\href{http://yongyeol.com/}{Yong-Yeol Ahn}, 
		\href{https://www.soic.indiana.edu/faculty-research/directory/profile.html?profile_id=266}{Christine Ogan},  
		\href{http://mypage.iu.edu/~weihuaan/}{Weihua An}
\end{innerlist}
%\vspace{3mm}

\href{http://www.ku.edu.tr/}{\textbf{Ko\c{c} University}},
Istanbul, Turkey
\begin{innerlist}
\item[] M.Sc.,
        \href{http://www.ku.edu.tr/}
             {Computer Science and Engineering},
        (July 2012) \\
	Thesis: Modal analysis of Myosin II and Identification of Functionally Important Sites \\    
	Advisors:
              \href{http://denizyuret.blogspot.com/}{Deniz Yuret}, 
              \href{http://home.ku.edu.tr/~akabakcioglu/}{Alkan Kabak\c{c}{\i}oglu}
\end{innerlist}
%\vspace{3mm}

\href{http://www.itu.edu.tr/}{\textbf{Istanbul Technical University}},
Istanbul, Turkey
\begin{innerlist}
\item[] B.Sc.,
        \href{http://www.ee.itu.edu.tr/}
             {Electronics Engineering}, (June 2010)     
\item[] B.Sc.,
        \href{http://www.fizik.itu.edu.tr/}
             {Physics Engineering}, (June 2012) 
\end{innerlist}


%%%%%%%% WORK %%%%%%%%

\section{Work Experience} 
%
\textbf{Microsoft Research}, Redmond WA, USA
\begin{innerlist}
\item[] Research Intern (June 2015 - September 2015): I worked in the CLUES group at MSR. I studied social media timelines of individuals to detect experiential activities and analyzed outcomes of those actions. This project involved analyzing search query logs and social media timelines.
\end{innerlist}
%\vspace{5mm}

\textbf{Microsoft Research}, Redmond WA, USA
\begin{innerlist}
\item[] Research Intern (June 2014 - September 2014): I worked in the CLUES group at MSR. I studied how individuals in social networks adopt their behaviors to match with their inner intents. I carried out experiments on Amazon Mechanical Turk platform to justify our hypothesis at the micro level and analyzed Twitter data to validate effects at the macro level.
\end{innerlist}
%\vspace{5mm}


\textbf{Stonefish Software Consultant}, Istanbul, Turkey
\begin{innerlist}
\item[] Software Developer (August 2009 - March 2010): I worked on an enterprise web application development using C\#, ASP.NET, and MSSQL. 
\end{innerlist}
%\vspace{5mm}

%\textbf{Istanbul Technical University Computer Center}, Istanbul, Turkey
%\begin{outerlist}
%\item[] Asistant Student (April 2008 - August 2009)
%\end{outerlist}
%\vspace{5mm}

%\textbf{ASELSAN}, Ankara, Turkey 
%\begin{outerlist}
%\item[] Intern (June 2009 - August 2009): I studied different forms of network programming for data transfer. I also worked on implementation of observation station platform for unmanned vehicles. 
%\end{outerlist}
%\vspace{5mm}

%\textbf{Istanbul Technical University Computer Center}, Istanbul, Turkey 
%\begin{outerlist}
%\item[] Intern (June 2008 - August 2009): I manufactured a temperature monitoring device using DS2019 sensor on Atmega16 microprocessor. Communication between microprocessor and server is implemented via RS232 protocol. Server and clients communicated through WPF on .NET framework.
%\end{outerlist}
%\vspace{5mm}

%\textbf{Control Avionic Laboratory}, Istanbul Technical University, Istanbul, Turkey
%\begin{outerlist}
%\item[] Intern (June 2007 - August 2007): I worked on improving distortion problem for the robotic testbed using MATLAB and OpenCV. 
%\end{outerlist}
%\vspace{5mm}


\section{Honors and Awards}
Invited to attend the $7^{th}$ Heidelberg Laureate Forum \\
University Distinguished Ph.D. Dissertation Award for Indiana University 2018 in the Social Sciences \\
Fragile Families Challenge top scoring submission for progress prize\\
Best paper award at Web Science Conference, 2014\\
Best poster award at Conference on Complex Systems, 2015\\
\underline{Travel grants}: SIGWEB for WebSci'16 (750\$), ICWSM'16 (350\$), ACM for COSN'14 (1,500\$), IU RKCSI for INFORMS'16 (500\$)\\
Research Assistantship Indiana University (2012-2016) \\
Scholarship from TUBITAK (equivalent of NSF in Turkey) (2010-2012)\\
%Graduate scholarship from Ko\c{c} University (2010-2012)\\
Chamber of Electrical Engineers of Turkey (EMO) Undergraduate Project Competition $1^{st}$ place (2010) \\
Bosch Scholarship for Undergraduate Education (2009)\\
%Istanbul Technical University Student Council Vice President (2009)
%Microsoft Student Partner (2008-2010)\\
%Euroskills 2010, Portugal, Mobile Robotic Expert\\
%Worldskills 2009, Canada, Mobile Robotic Competitor\\
%Euroskills 2008, Netherlands, 3rd rank in Mobile Robotic \\
\vspace{5mm}


%%%%%% PUBLICATIONS %%%%%%
\section{Publications  \\ 
{\footnotesize \href{https://scholar.google.com/citations?user=t8YAefAAAAAJ}{Google Scholar} \\ citations: 1,830 \\h-index: 15 \\ i10-index: 15}}
\textbf{Journal Articles}
\begin{innerlist}

\item[J.11] Yang K., \textbf{Varol O.}, Davis C., Ferrara E., Flammini A., Menczer F. \textbf{``Arming the public with AI to counter social bots''} \textit{Human Behavior and Emerging Technologies}, 2019

\item[J.10] R. Fan, \textbf{O. Varol}, A. Varamesh, A. Barron, I. Leemput, M. Scheffer, J. Bollen \textbf{``The minute-scale dynamics of online emotions reveal the effects of affect labeling"}, \textit{Nature Human Behavior},2018

\item[J.9] C. Shao, GL Ciampaglia, \textbf{O. Varol}, K. Yang, A. Flammini, F. Menczer \textbf{``The spread of low-credibility content by social bots''}, \textit{Nature Communications} 9:4787, 2018

\item[J.8] \textbf{O. Varol}, I. Uluturk, \textbf{``Deception Strategies and Threats for Online Discussions''} \textit{First Monday} 22(5), 2018

\item[J.7] \textbf{O Varol}, E Ferrara, F Menczer, A Flammini. \textbf{``Early Detection of Promoted Campaigns on Social Media''}. \textit{EPJ Data Science} 6(13), 2017

\item[J.6] C. Ogan, \textbf{O. Varol} \textbf{``What is gained and what is left to be done when content analysis is added to network analysis in the study of a social movement: Twitter use during Gezi Park''}, \textit{Information, Communication and Society} 20(8):1220-1238, 2017

\item[J.5] E Ferrara, \textbf{O Varol}, C Davis, F Menczer, A Flammini \textbf{``The Rise of Social Bots''}, \textit{Communications of the ACM} 59(7):96-104, 2016

\item[J.4] Davis CA, Ciampaglia GL, Aiello LM, Chung K, Conover MD, Ferrara E, Flammini A, Fox GC, Gao X, Gon�alves B, Grabowicz PA, Hong K, Hui P, McCaulay S, McKelvey K, Meiss MR, Patil S, Peli Kankanamalage C, Pentchev V, Qiu J, Ratkiewicz J, Rudnick A, Serrette B, Shiralkar P, \textbf{Varol O}, Weng L, Wu T, Younge AJ, Menczer F. \textbf{``OSoMe: The IUNI observatory on social media''}, \textit{PeerJ Computer Science}  2: e87, 2016

\item[J.3] V.S. Subrahmanian, A. Azaria, S. Durst, V. Kagan, A. Galstyan, K. Lerman, L. Zhu, E. Ferrara, A. Flammini, F. Menczer, R. Waltzman, A. Stevens, A. Dekhtyar, S. Gao, T. Hogg, F. Kooti, Y. Liu, \textbf{O. Varol}, P. Shiralkar, V. Vydiswaran, Q. Mei, T. Huang. \textbf{``The DARPA Twitter Bot Challenge''}, \textit{IEEE Computer} 49(6), 2016

\item[J.2] M JafariAsbagh, E Ferrara, \textbf{O Varol}, F Menczer, A Flammini \textbf{``Clustering memes in social media streams''}, \textit{Social Network Analysis and Mining} 4(237):1-13, 2014

\item[J.1] \textbf{O. Varol}, D. Yuret, B. Erman, A. Kabak\c{c}{\i}oglu \textbf{``Mode coupling points to functionally important residues in Myosin II''}, \textit{PROTEINS: Structure, Function, and Bioinformatics} 82(9):1777-�86, 2014

\end{innerlist}
\vspace{3mm}


\textbf{Refereed Conference Proceedings}

\textit{Conferences are a main venue of research dissemination in computer science. Conference proceedings are peer-reviewed and acceptance rates for top-tier conferences range from 10-15\%.}
\begin{innerlist}

\item[C.13] \textbf{Varol O.}, Davis C., Ferrara E., Menczer F., Flammini A. \textbf{``Online Human-Bot Interactions: Detection, Estimation, and Characterization''}, ICWSM'17

\item[C.12] Olteanu, A., \textbf{Varol, O.}, Kiciman, E. \textbf{``What Does Social Media Say about the Outcomes of Personal Experiences''}, CSCW'17

\item[C.11] Ferrara E., Wang W., \textbf{Varol O.}, Flammini A., Galstyan A. \textbf{``Predicting online extremism, content adopters, and interaction reciprocity''}, SocInfo'16

\item[C.10] \textbf{O Varol}, \textbf{``Spatiotemporal Analysis of Censored Content on Twitter''}. WebScience'16

\item[C.9] A Das, S Gollapudi, E Kiciman, \textbf{O Varol}, \textbf{``Information Dissemination in Heterogeneous-Intent Networks''}. WebScience'16

\item[C.8] E. Ferrara, \textbf{O Varol}, F Menczer, and A Flammini. \textbf{``Detection of Promoted Social Media Campaigns''}. ICWSM'16

\item[C.7] A Olteanu, \textbf{O Varol}, E Kiciman. \textbf{``Towards an Open-Domain Framework for Distilling the Outcomes of Personal Experiences from Social Media Timelines''}. ICWSM'16

\item[C.6] C Davis$^\dagger$, \textbf{O Varol}$^\dagger$, E Ferrara, A Flammini, F Menczer \textbf{``BotOrNot: A System to Evaluate Social Bots''}. WWW'16 Developers Day

\item[C.5] \textbf{O Varol}, E Ferrara, C Ogan, F Menczer, and A Flammini. \textbf{``Evolution of online user behavior during a social upheaval''}. WebScience'14 (\textbf{Best paper award}) %(Full paper, acceptance rate 14\%) 

%\textcolor{red}{*}
\item[C.4] \textbf{O Varol} and F Menczer. \textbf{``Connecting Dream Networks Across Cultures''}. WWW 2014 workshop on ``Connecting Online \& Offline Life'' (COOL)

\item[C.3] Ferrara, E., \textbf{Varol, O.}, Menczer, F. \& Flammini, A. \textbf{``Traveling Trends: Social Butterflies or Frequent Fliers?''}, ACM Conference on Online Social Networks (COSN 2013) %(Full paper, acceptance rate 15\%) 

\item[C.2]  E Ferrara, M JafariAsbagh, \textbf{O Varol}, V Qazvinian, F Menczer, A Flammini \textbf{``Clustering Memes in Social Media''}, ASONAM 2013 %(Full paper, acceptance rate 13\%)

\item[C.1] Yasemin Alban; Tuba Ayhan; \textbf{Onur Varol}; M��tak Erhan Yal\c{c}{\i}n \textbf{``A Feature Filtering Method for EEG Data Classification''}, Signal Processing and Communications Applications (SIU), IEEE 19th Conference, Antalya, April 20-22 2011.
\end{innerlist}

\vspace{3mm}
\textbf{Book chapter}
\begin{innerlist}

\item[B.1] \textbf{Varol O.}, Davis C., Menczer, F., Flammini, A. \textbf{``Feature Engineering for Social Bot Detection''}, Feature Engineering for Machine Learning and Data Analytics (2018): 311.
\end{innerlist}

\vspace{3mm}
\textbf{Patent}
\begin{innerlist}
\item[P.1] \textbf{``Peak sale and one year sale prediction for hardcover first releases''} Yucesoy, B., Wang, X., Barabasi, A. L., \textbf{Varol, O.}, Ruppert, P., Eliassi-Rad, T.  (2019). U.S. Patent Application No. 16/012,181.
\end{innerlist}


\vspace{3mm}
\textbf{Preprints \& Under review}
\begin{innerlist}
\item[U.3] \textbf{O. Varol}, I. Uluturk \textbf{``Journalists on Twitter: Self-branding, Audiences, and Involvement of Bots''}
\item[U.2] X. Wang, \textbf{O. Varol}, T. Eliassi-Rad \textbf{``L2P: Learning to Place for Estimating Heavy-Tailed Outcomes''}
\item[U.1] X. Wang, B. Yucesoy. \textbf{O. Varol}, T. Eliassi-Rad, AL. Barab�si \textbf{``Success in Books: Predicting Book Sales Before Publication''}
\end{innerlist}


\section{Conducted Research and Academic Experience}
\textbf{Research Projects}
\begin{innerlist}
\item[] \textbf{DOISAC}: Project name stands for ``Detecting Orchestrated Information and Synthetic Account Campaigns". This project founded by the Office of Naval Research aims at detecting orchestrated information and synthetic activity campaigns on social media using machine learning and computational tools. In this project, I studied individual and group activities of terrorist recruiters. We build predictive models to identify accounts with malicious intentions and activities.
			 
\begin{innerlist}        
        \item[] PIs:
        		 \href{http://www.emilio.ferrara.name/}
              	   {Emilio Ferrara}, 
              \href{https://sites.google.com/site/aflammin/}
                   {Alessandro Flammini},        
\end{innerlist}

\item[] \href{http://cnets.indiana.edu/groups/nan/despic}
			 {\textbf{DESPIC}}: Project name stands for ``Detecting Early Signature of Persuasion in Information Cascades" and aims to design a system detect persuasion campaigns at their early stage of inception, in the context of online social media. Our team built a system that analyzes social media data and extracts network, temporal, content, and user-based features to detect online campaigns. I worked on several modules of this framework: (i) a clustering procedure that uses metadata to compute similarity between memes; (ii) a classification system that determines whether a meme is potentially an orchestrated campaign or a genuine, grassroots conversation; (iii) a social bot detection framework called BotOrNot. This project founded by DARPA SMISC program.
			 
\begin{innerlist}        
        \item[] PIs:
              \href{https://sites.google.com/site/aflammin/}
                   {Alessandro Flammini}, 
              \href{http://cnets.indiana.edu/people/filippo-menczer}
              	   {Filippo Menczer}        
\end{innerlist}

\item[] \href{http://cnets.indiana.edu/groups/nan/truthy}
			 {\textbf{Truthy}}: This project aims to understand how information propagates through complex socio-technical information networks. In this project, I analyzed and studied roles of individuals during social upheavals, diffusion of trending topics in spatio-temporal space, and characterization of social media censorship and its effect on user behavior. 
			 
\begin{innerlist}        
        \item[] PIs:
              \href{https://sites.google.com/site/aflammin/}
                   {Alessandro Flammini}, 
              \href{http://cnets.indiana.edu/people/filippo-menczer}
              	   {Filippo Menczer}        
\end{innerlist}

%\item[] Research on Modal analysis of Myosin II and Identification of Functionally Important Sites: During my Master's studies in Koc University, I worked on analysis of protein fluctuations to identify functionally important residues as my thesis project.
%\begin{innerlist}        
%        \item[] Advisors:
%              \href{http://denizyuret.blogspot.com/}
%                   {Deniz Yuret},
%              \href{http://home.ku.edu.tr/~akabakcioglu/}
%              	   {Alkan Kabak\c{c}{\i}oglu}        
%        \end{innerlist}

%\item[] Research on Modeling of Social Networks and Phase Transitions of Complex Systems (Graduation project for B.Sc in Physics Engineering)
%\begin{innerlist}	
%        \item[] Advisor:
%              \href{http://web.itu.edu.tr/~erzan/}
%                   {Ay{s}e Erzan}        
%\end{innerlist} 
%\item[] Research on EEG Signal Processing and Classification (Graduation project for B.Sc in Electronics Engineering)
%\begin{innerlist}	
%        \item[] Advisor:
%              \href{http://web.itu.edu.tr/~yalcinmust/}
%                   	{M�stak Erhan Yal\c{c}{\i}n}        
%\end{innerlist} 

\end{innerlist}

%\textbf{Projects}
%\begin{outerlist}
%\item[] Ko\c{c} University, Istanbul, Turkey
%\begin{innerlist}
%\item Community networks and opinion dynamic models (2012)
%\item Designed and implemented a sketch based map builder in 3D graphics (2011)
%\item Designed and programmed a \href{http://tourguide.vypro.org/}{Tourist Guidance Multi Model Interface Application}(2011) 
%\item Implemented unsupervised part-of-speech tagger, regression-based statistical machine translation decoder, spam filter using language models in Machine Learning Course (2011)  
%\end{innerlist}

%\item[] Istanbul Technical University, Istanbul, Turkey
%\begin{innerlist}
%\item EEG Signal Processing and Classification for Mobile Robot Navigation (2010)
%\item Scanning Tunnel Microscope control electronic design and software implementation (2010)
%\item Human machine interface for mouse control using hand gestures (2009) 
%\item EKG device electronic and software implementation (2009) 
%\end{innerlist}
%\end{outerlist}

\vspace{3mm}
\textbf{Teaching Experience}
%
\begin{innerlist}
\item[] \href{https://www.northeastern.edu/}{\textbf{Northeastern University}},
Boston MA, USA
\begin{innerlist}
\item[] Data Mining Techniques (Guest Lecturer, Fall 2018)
\end{innerlist}
\end{innerlist}

\begin{innerlist}
\item[] \href{https://www.iu.edu}{\textbf{Indiana University}},
Bloomington IN, USA
\begin{innerlist}
\item[] Topics in Informatics: Performance Analytics (Teaching Assistant, Spring 2017)
\end{innerlist}
\end{innerlist}
%
\begin{innerlist}
\item[] \href{http://www.ku.edu.tr}{\textbf{Ko\c{c} University}},
Istanbul, Turkey
\begin{innerlist}
\item[] Machine Learning (Teaching Assistant, Spring 2012)
\item[] Microprocessors (Teaching Assistant, Fall 2011)
\item[] Probability and  Random Variables (Teaching Assistant, Spring 2011)
\item[] Discrete Mathematics (Teaching Assistant, Fall 2010)
\end{innerlist}
\end{innerlist}

\vspace{3mm}
\textbf{Mentoring}
\begin{innerlist}
\item[] \href{https://www.northeastern.edu/}{\textbf{Northeastern University}},
Boston MA, USA
\begin{innerlist}
\item[] Xindi Wang (NetSI Phd student, [Spring 2017,---])
\end{innerlist}
\end{innerlist}

\begin{innerlist}
\item[] \href{https://www.iu.edu}{\textbf{Indiana University}},
Bloomington IN, USA
\begin{innerlist}
\item[] Mohsen Sayyadi (CS PhD student, [Spring 2018,---])
\item[] Kaicheng Yang (Informatics PhD student, [Spring 2017,---])
\item[] Zoher Kachwala (CS MSc student, Spring 2018)
\item[] Shradha Baranwal (CS MSc student, Spring 2018)
\end{innerlist}
\end{innerlist}


%\newpage
\section{Talks and Events}
\textbf{Invited Talks}
\begin{innerlist}
\item Modeling individual and group behavior in complex interactive systems, University of Pittsburgh School of Computing and Information (04/17/2019)
\item Detecting Social Bots and Modeling Online Behavior, Microsoft Research AI Breakthroughs, Redmond, USA (09/17/2018)
\item Observatory of Social Media, Exploring Media Ecosystems Conference at MIT Media Lab (03/06/2017)
\item Predicting adolescence academic performance through panel survey data, Fragile Families Challenge Scientific Workshop at Princeton University (11/17/2017)
\item Observatory on Social Media: Investigating Bots and Disinformation, Digital Disinformation Forum Stanford (06/26/2017)
\item The Impact of Censorship on Tweeting Behaviors, Indiana University Conference on Big Data and Network Science (03/23/2017)
\item Studying Individuals and Groups using Online Data, Northeastern University Network Science Institute (03/09/2017)
\item Analysis of Online Discourse and Information Diffusion, INFORMS Meeting Nashville (11/15/2016)
\item Detection of Online Manipulation, UND Big Data Summit Event (04/07/2016)
\item Twitter Applications: Industry Panel Speaker, UND Big Data Summit Event (04/07/2016)
\item Campaign and Social Bots Detection on Social Networks, Workshop in Network Science (WINS) Indiana University (02/18/2016)
\item Evolution of online user behavior during a social upheaval, Indiana University Turkish Flagship Center (01/21/2015)
\item Studying Social Dynamics Through Social Media, Istanbul Technical University Physics Department Colloquia (04/28/2014)
\end{innerlist}

\textbf{Conference Presentations}
\begin{innerlist}
\item Detection, Estimation, and Characterization of Bot Nodes in Social Networks, Network Science Conference, Indianapolis USA (06/21/2017)
\item Distilling the Outcomes of Personal Experiences from Social Media Timelines, International Conference on Computational Social Science, Evanston Illinois USA (06/25/2016)
\item Information Dissemination in Heterogeneous-Intent Networks, Web Science Conference, Hannover Germany (05/24/2016)
\item Evolution of Online User Behavior During Social Upheaval, Conference on Complex Systems (CCS), Tempe Arizona USA (09/27/2015)
\item Connecting Dream Networks Across Cultures, COOL Workshop at World Wide Web (WWW) Conference, Seoul Korea (04/08/2014)
\end{innerlist}

%\textbf{Conference Presentations}
% Netsci,ic2s2, websci

%\textbf{Workshop \& Schools}
%\begin{outerlist}
%\item A Workshop on the Allosteric Mechanisms in Protein Regulation, Ko� University (July 2012)
%\item Computational Methods for Life Sciences and Nanotechnology, Ko� University (January 2012)
%\item Phase Transition and Renormalization Groups, Feza G�rsey Institute (August 2010)
%\item International Summer School and Research Workshop on Complexity, Feza G�rsey Institute - Imperial College (September 2011)
%\end{outerlist}
%\vspace{5mm}

%%%%%%%%%%%%%%%%
%\section{Test Scores}
%\textbf{GRE:} A:168, V:138\\
%\textbf{TOEFL IBT:} 84\\


\section{Computer and Language Skills}
\textbf{Programming Languages and Skills:}
\begin{innerlist}
\item[] Frequent user of Python for data analysis using Matplotlib, NetworkX, Pandas, Scikit-learn, etc.
\item[] Experience in \LaTeX{}, C / C++, C\#, Java, MATLAB, R, OpenBUGS
\item[] Familiar with HTML, CSS, JS for frontend, Flask, Django, and ASP.NET for larger applications
\item[] Used MySQL, NoSQL (CouchDB and Riak), Map-Reduce
\item[] View projects on Github: \href{https://github.com/onurvarol}{github.com/onurvarol}
%\item[] OpenCV, OpenGL 
%\item[] Cisco Routing
\end{innerlist}

\textbf{Languages:}
\begin{innerlist}
\item[] English (fluent), German (beginner), Turkish (native)
\end{innerlist}

\section{Community Service}
\textbf{Journal reviewer}: Nature Scientific Reports; Nature Machine Intelligence; EPJ Data Science; PLoS One; Network Science; Social Networks; Journal of Computational Social Sciences; Information, Communication and Society; PeerJ-CS; ACM Transactions on Social Computing; ACM Transactions on the Web; IEEE Transactions on Knowledge and Data Engineering; IEEE Access; IEEE Intelligent Systems; IEEE Communications Magazine; Big Data; First Monday; Journal of Medical Internet Research; Journal of the Royal Society Interface; Computers in Human Behavior; Data Mining and Knowledge Discovery; Information; Online Information Review;  Social Science Research; Social Science Computer Review; Online Social Networks and Media; International Journal of Information Management \\
\textbf{Journal subreviewer}: Science Advances; Nature Communications \\
\textbf{Conference Senior PC member}: HyperText'18 \\
\textbf{Conference PC member}: WWW'17/'18/'19; WSDM'19; ICWSM'17/'18/'19; KDD'19; SDM'19; WebSci'18/'19; CHI'18; CSCW'18/'19; CompleNet'18; NetSci'17/'18; SocInfo'17/'18/'19; IC2S2'17/'18/'19; HyperText'17/'18/'19; DSAA'17/'18/'19\\
\textbf{Workshop co-chair}: Open Science for an Open Society (CCS'16)\\
\textbf{Workshop PC member}: IEEE BigData'17/'18; SIGIR Demo'18;\\
\textbf{Conference subreviewer}: ICWSM'15; IC2S2'15; ASONAM'15; WebScience'14

%\section{Applicable Coursework}
%\textbf{Informatics PhD with Statistics minor}
%\begin{innerlist}
%\item[] Introduction to Complex Systems
%\item[] Mathematical Modeling of Complex Systems
%\item[] Models in Cognitive Science
%\item[] Bayesian Theory and Data Analysis
%\item[] Nonparametric Theory and Data Analysis
%\item[] Statistical Methods for Causal Inference
%\item[] Model Comparison and Selection
%\item[] Time Series Analysis
%\end{innerlist}
%
%\textbf{Computer Science MS}
%\begin{innerlist}
%\item[] Machine Learning
%\item[] Information Retrieval
%\item[] Linear System Theory
%\item[] Network Models and Optimization
%\end{innerlist}
%
%To obtain list of all the courses that I enrolled, you can visit my \href{http://www.onurvarol.com/courses.html}{courses page} (onurvarol.com/courses.html)

\section{Interests}
I enjoy playing basketball \emoji{basketball}, foosball \emoji{foosball}, and backgammon \emoji{dice}. I also love traveling \emoji{travel} a lot and feeling wanderlust, you can visit my \href{http://www.onurvarol.com/my\_travels/}{online travel journal} (onurvarol.com/my\_travels).

%\section{References}
%\textbf{Dr. Albert-L\'{a}szl\'{o} Barab\'{a}si} --- alb@neu.edu (Postdoctoral advisor)\\
%Robert Gray Dodge Professor of Network Science, Northeastern University Boston, MA\\
%\textbf{Dr. Filippo Menczer} --- fil@indiana.edu (Doctoral advisor)\\
%Professor at the School of Informatics and Computing, Indiana University Bloomington, IN \\
%\textbf{Dr. Emilio Ferrara} 
%\textbf{Dr. Tina Eliassi-Rad} 
%\textbf{Dr. Alessandro Flammini} --- aflammin@indiana.edu\\
%Professor at the School of Informatics and Computing, Indiana University Bloomington, IN 

\begin{flushright}
	Last Updated: \today
\end{flushright}

\end{document}
%%%%%%%%%%%%%%%%%%%%%%%%%% End CV Document %%%%%%%%%%%%%%%%%%%%%%%%%%%%%